\chapter*{Appendix A: Kriging meta model}

The Kriging metamodel has already been briefly introduced in chapter \ref{ch:4} but here we want to talk a little bit more on the numerical procedure behind it and also present some implementation example.

The Kriging method was first derived in for make prediction of missing geostatics data (\citet{krige1951statistical}). However this methodology has been further generalized and applied extensively in recent literature as meta model.
The method can treat highly non linear output and can be used to either interpolate or extrapolate response.

After a exploration of the design of the experiment the data produced is usually organized as $(\mathbf{x_i}, y(\mathbf{x_i}))$  $i=1,...,n$ where
\begin{itemize}
	\item $\mathbf{x_i}$ is the i-th element of the input of the experiment. It can contain multiple parameters, of length $k$
	\item $y_i$ is the scalar response of the experiment for the vector of inputs $\mathbf{x_i}$
\end{itemize}

The Kriging response for a new generic set of input $\boldsymbol{\chi}$ is given by:
\begin{equation}
	\hat{f}(\boldsymbol{\chi}) = \sum_{i=1}^{N} \lambda_i(\mathbf{x}) f(\mathbf{x_i}) =  \sum_{i=1}^{N} \lambda_i(\mathbf{x}) y_i
\end{equation}

The weights $\lambda_i$ are chosen to ...
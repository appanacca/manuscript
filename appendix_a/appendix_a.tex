\chapter*{Appendix A: Kriging meta model}

The Kriging metamodel has already been briefly introduced in chapter \ref{ch:4} but here we want to talk a little bit more on the numerical procedure behind it and also present some implementation example.

The Kriging method was first aimed to make prediction of missing geostatics data (\citet{krige1951statistical}). However this methodology has been further generalized and applied extensively in recent literature as metamodel for large variety of experiments.
The method can treat highly non linear output and can be used to either interpolate or extrapolate response from a sample design set.

In this discussion the $\hat{f}(\boldsymbol{\chi})$ is a model for the function $f(\boldsymbol{\chi})$ and $\hat{y}$ is the model prediction of the true response $y = f(\boldsymbol{\chi})$ that is evaluated at the point $\chi$. $n$ is the number of point in the sample design set and $k$ is the number of input of the experiment.

After the exploration of the design possibilities the database produced is usually organized as $(\mathbf{x_i}, y(\mathbf{x_i}))$  $i=1,...,n$ where
\begin{itemize}
	\item $\mathbf{x_i}$ is the i-th vector element containing the $k$ input parameters for the i-th experiment run
	\item $y_i$ is the scalar response of the experiment for the vector of inputs $\mathbf{x_i}$ \footnote{$y_i$ is always a scalar because even in case of multiple output for an experiment run they are supposed to be uncorrelated. It means that if we had $p$ elements in each $\mathbf{y_i}$ will have to build $p$ metamodels}
\end{itemize}

The Kriging response for a new untried input point $\boldsymbol{\chi}$, containing $k$ elements, is given by the linear \textit{predictor}:
\begin{equation}
\hat{y} = \hat{f}(\boldsymbol{\chi}) = \sum_{i=1}^{N} \lambda_i(\mathbf{x}) f(\mathbf{x_i}) =  \sum_{i=1}^{N} \lambda_i(\mathbf{x}) y_i
\end{equation}

$\hat{y}$ is considered to be a new realization of the random Gaussian process that has generated the set $S_y = [y_1,...,y_n]$.
The weights $\lambda_i$ are the solution of a linear system obtained by minimizing the variance of the error between the predictor and the random process.
The best \textit{linear unbiased predictor} BLUP is so obtained finding the weights $\lambda_i$ that minimize:

\begin{equation}
MSE[\hat{y}(\chi)] = E \left[\left(\lambda(\chi)y -y(\chi)\right)^2\right]
\label{eq:var_err}
\end{equation}

under the unbiasedness condition:

\begin{equation}
E \left[ \lambda(\chi)y -y(\chi)  \right]
\end{equation}

fai tutta la procedura per calcolare lambda

poi dici le varie scelte per la funzione di correlazione 

ti conviene parlare del periodgramm ? NO aggiungeresti troppa carne al fuoco forse..?




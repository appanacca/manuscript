\chapter*{Abstract}

Any natural surface is in essence non-smooth, consisting of more or less regular roughness and/or mobile structures of different scales. From a fluid mechanics point of view, these natural surfaces offer better aerodynamic performances when they cover moving bodies, in terms of drag reduction, lift enhancement or control of boundary layer separation; this has been shown for boundary layer or wake flows around thick bodies. The numerical simulation of microscopic flows around "natural" surfaces is still out of reach today. Therefore, the goal of this thesis is to study the modeling of the apparent flow slip occurring on this kind of surfaces, modeled as a porous medium, applying Whitaker's volume averaging theory. This mathematical model makes it possible to capture details of the microstructure while preserving a satisfactory description of the physical phenomena which occur. 

The first chapter of this manuscript provides an overview of previous efforts to model these surfaces, detailing the most important results from the literature. The second chapter presents the mathematical derivation of the volume-averaged Navier-Stokes equations (VANS) in a porous medium. In the third chapter the flow stability at the interface between a free fluid and a porous medium, formed by a series of rigid cylinders, is studied. The presence of this porous layer is treated by including a drag term in the fluid equations. It is shown that the presence of this term reduces the rates of amplification of the Kelvin-Helmholtz instability over the whole range of wavenumbers, thus leading to an increase of the wavelength of the most amplified mode. In this same context, the difference between the isotropic model and a tensorial approach for the drag term was evaluated, to determine the most consistent approach for the study of the stability of this type of flows. This has led to the conclusion that the most relevant model is the one using the apparent permeability tensor. In the following chapter, based on this last result, the apparent permeability tensor, based on over one hundred direct numerical simulations carried out over microscopic unit cells, has been identified for a three-dimensional porous medium consisting of rigid cylinders. In these configurations the tensor varies according to four parameters: the Reynolds number, the porosity and the orientation of the average pressure gradient (the latter defined by two Euler angles). This parameterization makes it possible to capture local three-dimensional effects. This database has been set up to create, based on a kriging-type approach, a behavioral meta-model for estimating all the components of the apparent permeability tensor.

In the fifth chapter, simulations of the VANS equations are carried out on a macroscopic scale after the implementation of the meta-model, and this allows reasonable computing times. The validation of the macroscopic approach is performed on a flow in a closed cavity covered with a porous layer and a comparison with the results of a very precise DNS, homogenized a posteriori, shows a very good agreement and demonstrates the relevance of the approach. The next step has been the study of the passive control of the separation of the flow past a hump which is placed on a  porous wall, by the same macroscopic VANS approach.  Finally, general conclusions and possible directions of research in the field are presented in the last chapter.
\chapter*{Résumé}

Le nombre de caractères ne doit pas être supérieur à 4000. 
Les résumés français et anglais doivent tenir tous deux en 4ème de couverture de votre manuscrit, et les résumés déposés ici doivent être identiques à ceux du manuscrit.
Presque toutes les surfaces existant dans la Nature ne sont pas lisses, elles ont des rugosités plus ou moins régulières et/ou des structures mobiles à différentes échelles
Ces surfaces ont des meilleures performances aérodynamiques (réduction de traînée, augmentation de la portance, contrôle de séparation…) et cela a été prouvé sur des cas comme des couches limites et des corps épais.

Faire des calculs à l’échelle microscopique n’est pas faisable, de ce fait nous voulons trouver une manière de modéliser le glissement apparent sur ces surfaces (modélisé comme un milieu poreux) utilisant la théorie de Moyenne-Volumique de Whitaker [134].
Ce modèle mathématique nous permet de représenter en moyenne les détails de la micro-structure sur ces surfaces mais cela donne encore une bonne description des phénomènes.
Le premier chapitre de ce manuscrit donne un panoramique des efforts précédents sur la modélisation de ces surfaces, et présente les indications plus importantes qui sont tirées de la bibliographie.

Le deuxième chapitre présente la dérivation mathématique des équations de Navir-Stokes Moyenne en Volume (VANS) dans un milieu poreux.
Dans le troisième chapitre nous étudions la stabilité de l’écoulement à l’interface entre un fluide libre et un milieu poreux, faite d'une série de cylindres rigides.
La présence de cette couche poreuse a été traitée avec un terme de traînée dans les équations de fluides.
On montre que la présence de ce terme reduit les facteurs d’amplification de l’instabilité des Kelvin-Helmholtz sur toute la gamme de nombre d’ondes et ainsi augmente la longueur d’onde du mode le plus amplifie.
Dans le même contexte on a calculé la différence entre un modèle isotrope de traînée et une approche tensorielle pour le terme de traînée, dans le but de comprendre qu’elle était l’approche plus solide pour le calcul de stabilité. Notre conclusion était que le modèle qu'utilise le tenseur de perméabilité apparent est le meilleur choix.
Basé sur ce résultat, dans le chapitre quatre nous avons calculé numériquement le tenseur de perméabilité apparent, pour un milieu poreux 3D constitué de cylindres rigides, avec les problèmes de fermeture dérivées avec les VANS. 
En effet ce tenseur vari avec le nombre de Reynolds, l’orientation du gradient de pression moyenne est la porosité. 
Les deux angles d’Euler qui définissent l’orientation du gradient de pression sont inclus dans le modèle pour capturer le possible effet 3D.  Partant de cette base de données, une interpolation de kriging a été utilisée pour obtenir une estimation de tous les composants du tenseur pour chaque paramètre d’entrée.

Dans le dernier chapitre nous montrons des résultats basés sur ce méta modèle; l’utilisation de ce modèle d’ordre réduit avec un code numérique basé sur les équations VANS à l’échelle macroscopique, permet de maintenir les temps de calcul entre des niveaux raisonnable.
Pour validé cette description macroscopique nous comparons la simulation d'une cavité ferme avec une couche poreuse, avec les VANS et avec les résultats d’un DNS homogénéisé a posteriori. 
Cette comparaison montre un bon accord entre les deux modèles.
Avec la même description macroscopique nous étudions la possibilité d’utiliser les équations VANS pour faire des simulations d’écoulements détachés et constater si cette couche poreuse peut contrôler cette séparation.
Enfin, le dernier chapitre présente la conclusion de ce travail et la possible direction de recherche dans ce domaine.
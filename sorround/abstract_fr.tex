\chapter*{Résumé}

Toute surface naturelle est par essence non lisse, elle est constituée de rugosités plus ou moins régulières et / ou de structures mobiles d’échelles différentes et variées. D’un point de vue mécanique des fluides, ces surfaces naturelles proposent des meilleures performances aérodynamiques en termes de réduction de traînée, d’augmentation de la portance ou de contrôle du décollement lorsqu’elle couvre des corps en mouvement ; et cela a été prouvé pour des écoulements de couches limites ou de sillage, autour de corps épais. La simulation numérique d’écoulements aux échelles microscopiques autour des surfaces « naturelles » demeure de nos jours encore hors de portée. En conséquence, la thèse a pour objet d’étudier la modélisation du glissement apparent de l’écoulement sur ce genre surface, modélisée comme un milieu poreux, appliquant la théorie de la moyenne-volumique de Whitaker. Ce modèle mathématique permet globalement de représenter les détails de la micro-structure de ses surfaces en moyenne, tout en conservant une description satisfaisante des phénomènes physiques induits par l’écoulement. 
Le premier chapitre de ce manuscrit dresse un panorama des efforts antérieurs portant sur la modélisation de ces surfaces en précisant les résultats les plus importants issus de la littérature. Le deuxième chapitre présente la dérivation mathématique des équations de Navier-Stokes en moyenne volumique (VANS en anglais) dans un milieu poreux. Dans le troisième chapitre est étudiée la stabilité de l’écoulement à l’interface entre un fluide libre et un milieu poreux, formé par une série de cylindres rigides. La présence de cette couche poreuse est traitée par un terme de traînée dans les équations du fluide. On montre que la présence de ce terme réduit les taux d’amplification de l’instabilité de Kelvin-Helmholtz sur toute la gamme des nombre d’onde et ainsi augmente la longueur d’onde du mode le plus amplifié. Dans ce même contexte a été calculée la différence entre un modèle isotrope et une approche tensorielle pour le terme de traînée, afin de déterminer l’approche la plus consistante pour une étude de stabilité de ce type d’écoulement. Cela a mené à la conclusion que le modèle le plus pertinent est celui utilisant le tenseur de perméabilité apparent. Dans le chapitre suivant, en s’appuyant sur ce dernier résultat, le tenseur de perméabilité apparent est identifié sur la base d’une centaine de simulations numériques directes, pour un milieu poreux tridimensionnel constitué de cylindres rigides, où le problème de fermeture est abordé par la méthode VANS. Dans ces configurations ce tenseur varie en fonction de quatre paramètres : le nombre de Reynolds, la porosité et l’orientation du gradient moyen de pression (définie par deux angles d’Euler). Cette paramétrisation permet de capturer les effets tridimensionnels locaux. Cette base de données ainsi constituée a permis de créer, sur la base d’une approche de type kriging, un méta-modèle comportemental pour estimer toutes les composantes du tenseur de perméabilité apparente.


Dans le cinquième chapitre sont menées des simulations des équations VANS à l’échelle macroscopique après implémentation du méta-modèle qui autorise des temps de calcul raisonnables.  La validation de l’approche à l’échelle macroscopique est effectuée sur un écoulement dans une cavité fermé couverte d’une couche poreuse et une comparaison avec les résultats d’un DNS très précise, homogénéisés a posteriori montre un très bon accord et démontre la pertinence de la démarche. L’étape suivante a consisté en l’étude du contrôle du décollement pour un écoulement autour d’une bosse sur une paroi poreuse par cette même approche VANS macroscopique. Enfin des conclusions générales et des directions de recherche possibles dans le domaine d’étude sont présentées dans le dernier chapitre.
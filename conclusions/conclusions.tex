\chapter{Conclusions, recommendation and discussions}



\section{Main conclusion}

In chapter 1 we have reviewed the latest advances and open questions present in literature. The same chapter form a new and improved basis from which
 many researchers could find and/or explore new research path and ideas. In this final chapter we list the main results and conclusion that had been drawn 
 from our work.

\begin{itemize}

\item The volume averaged method has been detailed in key hypothesis. The mathematical procedure needed to find the macroscopic equations, and the closure 
problem, has also been presented. Some of the most notabale new extensions to the method has been included in the discussion that we hope it could help new  
researchers to approach the study of this method and serve as a review for who is already accostumed to it.

\item The sensitivity analysis shows that the VANS approach is the less sensitive one in respect to variations in the base flow. Also, the stability results
 still agree with the experimental results. One of the possible drawbacks in the use of the VANS model is the need of computing the effective permeability 
 tensor $\mathbf{H}$ although, even the parametrization of the drag force is not an easy task. The computational cost and difficulties to compute the 
 components of the permeability tensor is the main reason that have take us in the development of a metamodel for the tensor $\mathbf{H}$ in chapter 4.
 
\item Opposed to the result in the work of \citet{lasseux} for rectangular fibers, our computed effective permeability tensor for circular fibers is diagonal.
 This means that the geometrical shape of the porous structure is very important in this sense. A possible generalization of different porous structures is 
 shown in \citet{pauthenet} even though further investigations on the non diagonal terms are needed.

\item We have also shown that the 3D effects can be very important in changing the oermeability tensor components. In our data analysis, in the chapter 4, 
we have shown that the angle $\phi$ \footnote{the angle between the forcing term in the momentum equations and the fiber axis} have a big influence in the 
tensor $\mathbf{H}$ components, especially in the inertia regime.
The same angle $\phi$, when present, make the flow three dimensional and it bends the fluid path along the fiber axis. This process translate into a non zero
deviation angle $\gamma$ in the fiber axis direction.

\item In chapter 4 we have shown that the $\mathbf{H}$ metamodel has been developed up to a Reynolds number equal to $100$. This limit has been derived 
from the data and it was not fixed a priori. To estimate this limit we have checked the the direct comparison of $F^m$ and $F^M$ since it is a fair 
estimation of the correctness of the hypothesis behind the our closure model. For our geometry, and range of porosities tested, the correct limit is around 
Reynolds number $100$. After this limit the error between the two quantities start to be appreciable and so the closure problem \eqref{eq:M_problem} is no more
 correct. We suppose that at higher Reynolds number the liner correlation hypothesis between the averaged and perturbation (equations \eqref{eq:closure_viab_h1} and \eqref{eq:closure_viab_h2}) does not hold.

\item The interface treatment, based on the penalization method, has been investigated in chapter 5. It has been showed that the double linear
 smoothing of either the porosity and the permeability has a positive effect on the correctness of the homogenized model. We have also shown that the 
 linear porosity profile derive directly from the geometry of the porous media and it is exact. On the contrary the linear smoothing for the permeability tensor is 
 purely euristic, but it can be supported by the fact that the porosity effects are largerly the most important in the variability of $\mathbf{H}$. So is 
 possible to argue that the two fields should have the same interface treatment. Another confirmation for this fact came from the metamodel that if left "free" 
 it return the linear profile at the interface without previously imposing it.
 
 \item Our VANS approach has been tested in geometry that naturally develop spearation. THe inclusion of a porous media layer has been tested and the solver
 had shown good computational performance and no convergence problems has been found. However, the physics of the separation is no mush modified by the porous 
 layer. This findings suggest that the laminar suppression mechanisc could be not as effective as the turbulent one (already observed in literature). In any case more simulations
 with different parameters and different problem geometry are needed to generalize the results.

\item The OpenFoam implementation of the macroscopic solver based on the VANS equations can be downloaded from github from the address: \url{https://github.com/appanacca/porous_solvers_OF}.
The code listing is not directly showed in the manuscript since detailing the solver implementation would have require to explain lot of OpenFoam library 
technicalities. These details has been already addressed in multiple sources (\citet{jasak1996error}, \citet{moukalled2016finite} and 
\cite{maric2014openfoam}) and they are out of the scope of this work.
Although, to someone not new to OpenFoam programming the comments inside the code listing are sufficient to clarify the technical points.


\end{itemize}


\subsection{Possible future research extensions}

\begin{itemize}

\item The database from which we have built our metamodel for the tensor $\mathbf{H}$ can be extended. For example we could easily included more data 
points to make the model more reliable. Another the interesting part could be the extension to other fiber geometry section or even other completely different
 porous media geometries (shperes, rocks ...). The database could also been extended to moving porous media, the imput parameters could include some of the 
 typical dynamic parameters like the the mass ratio and or the stiffness of the fibers. New metamodelling approaches could also be explored, expecially as 
 the database grow \textit{deep neural networks} could permorf better than Kriging.

\item The validation of the the interface treatment need more data from DNS simulations or experiment in similar configurations. The availability of high 
resolution data is still a missing piece in the field.

\item The application of the macroscopic model to separated flow is only a starting point. We have shown that our model is capable of provide fairly correct 
homogenized flow field at a low computational cost. Although the capacity of a porous media layer to suppress the separation is still questioned, at least 
with the parameters used. This means that the optimal parameters are still to be found. An optimization procedure using the adjoint equation could solve this
 problem, now that we have clarified the penalization approach used in the VANS equations.

\item Another possible extension to the metamodel could be the implementation of the algorithm \ref{algo:vans_elastic} for elastic porous media. Since the 
VANS solver is already been implemented this extension using a Bernoulli beam should be fairly easy.
\end{itemize}

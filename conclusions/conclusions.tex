\chapter{Conclusions, recommendation and discussions}



\section{Main conclusion}

In chapter 1 we have collected most of the advances and open questions present in literature. The same chapter form a basis from which many researchers could find and/or explore new research path and ideas. In this section we recapitulate the main results and conclusion from our work.

\begin{itemize}


\item The volume averaged method has been detailed in his aspects and the key hypothesis has been stated. Also some of the new extensions of the method has been included in the discussion that we hope we could help new researchers that approach the study of this method.

\item The sensitivity analysis shows that the VANS approach is less sensitive to variations in the base flow. Also, the stability results still agree with the source drag term approach. One of the possible drawbacks in the use of the VANS model is the need of computing the effective permeability tensor $\mathbf{H}$ although, even the parametrization of the drag force is not an easy task, as stated in the first chapter. This is the main reason that have take us in the development of a metamodel for $\mathbf{H}$ in chapter 4.

\item Opposed to the result of \citet{lasseux} for rectangular fibers, our computed effective permeability tensor for circular fibers is diagonal. This means that geometrical shape parameters of the porous structure are very important in this sense. A possible generalization of different porous structures is shown in \citet{pauthenet} even though further investigations on the non diagonal terms are needed.

\item In chapter 4 we have shown that the $\mathbf{H}$ metamodel was developed till a Reynolds number equal to $100$. This limit has been derived from the data and it was not fixed a priori. As a matter of fact we have seen how the direct comparison of $F^m$ and $F^M$ is a fair estimation of the correctness of the hypothesis behind the our closure model. For our geometry and range of porosities tested the correct limit is around Reynolds number $100$. After that the error between the two quantities start to be appreciable and so the closure problem \eqref{label} is no more correct. We suppose that at higher Reynolds number the liner correlation hypothesis between the averaged and perturbation (equation \eqref{label}) is the one that does not hold.

\item The interface treatment, based on the penalization method, has been investigated and cleared in chapter 5. It has been showed that the double linear smoothing of either the porosity and the permeability has a positive effect on the correctness of the homogenized model. We have also shown that the linear porosity profile derive directly from the geometry of the porous media and it is exact.

\item The OpenFoam implementation of the macroscopic solver based on the VANS equations can be downloaded from github from the address: \url{https://github.com/appanacca/porous_solvers_OF}.
The code listing is not directly showed in the manuscript since detailing the solver implementation would have require to explain lot of OpenFoam library technicalities. These details has been already addressed in multiple sources (\citet{jasak1996error}, \citet{moukalled2016finite} and \cite{maric2014openfoam}) and they are out of the scope of this work.
Although, to someone not new to OpenFoam programming the comments inside the code listing are sufficient to clarify the technical points.
\end{itemize}


\subsection{Possible future research extensions}



The database from the which we have built our metamodel can be extended. For example we could easily included more data points to make the model more reliable. But the interesting part could be the extension of the same to other fiber geometry section or even other porous media geometries.
The database could also been extended to moving porous media, including some typical dynamic parameters like the the mass ratio and or the stiffness of the fibers.
Also novel metamodelling approaches could be explored as the database grow, like \textit{deep neural networks} that seem the work horse for very large databases (at the moment).

The validation of the the interface treatment need more data from DNS simulations or experiment in similar configurations. The availability of high resolution data is still a missing piece in the field.

The application of the macroscopic model to separated flow is only a starting point. We have shown that our model is capable of provide fairly correct homogenized flow field at a low computational cost. Although the capacity of a porous media layer to suppress the separation is still questioned, at least with the parameters used. This means that the optimal parameters are still to be found. An optimization procedure using the adjoint equation could solve this problem, now that we have clarified the penalization approach used in the VANS equations.

Of course the another possible extension to the metamodel could be the implementation of the algorithm \ref{algo:vans_elastic} for elastic porous media. Since the VANS solver is already been implemented this extension using a Bernoulli beam should be fairly easy.

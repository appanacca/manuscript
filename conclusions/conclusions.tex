\chapter{Conclusions, recommendations and discussions}

\chapquote{The question of whether Machines Can Think… is about as relevant as the question of whether Submarines Can Swim.}{Edsger Dijkstra}

\section{Main conclusions}

In chapter 1 we have reviewed the latest advances and open questions present in the literature. The same chapter is an attempt to produce a new and improved basis from which many researchers could find and/or explore new research paths and ideas. In this final chapter we list the main results and conclusions that can be drawn from the present work.

%\begin{itemize}

The volume average method has been detailed in its key assumptions. The mathematical procedure needed to find the macroscopic equations, and the closure problem, has also been presented. Also some of the most notable new contributions to the method have been included in the discussion. % that we hope could help new researchers to approach the study of this method.

 The sensitivity analysis shows that the VANS approach is the less sensitive one with respect to variations in the base flow. Also, the stability results agree reasonably with the experimental results. One of the possible drawbacks in the use of the VANS model is the necessity to compute the effective permeability tensor $\mathbf{H}$. %, although even the parametrization of the drag force is not an easy task.
The computational cost and the difficulties to compute the components of the permeability tensor are the main reason that have taken toward the development of a metamodel for the tensor $\mathbf{H}$ in chapter 4.
 
Opposed to the results in the work of \citet{lasseux} for rectangular fibers, our computed effective permeability tensor for circular fibers is, with a good approximation, diagonal. It means that the geometrical shape of the porous structure is very important for the characterization of the tensor structure. A possible generalization of different porous structures is shown in \citet{pauthenet} even though further investigations on the non diagonal terms are needed.

We have also shown that the 3D effects can be very important in changing the permeability tensor components. In our data analysis, in chapter 4, we has shown that the angle $\phi$\footnote{the angle between the forcing term in the momentum equations and the fiber axis.} has a large influence in the tensor $\mathbf{H}$ components, especially in the inertia regime.
The same angle $\phi$ makes the flow three dimensional and it bends the fluid path along the fiber axis. This process translates into a non-zero deviation angle $\gamma$ in the fiber axis direction.

In chapter 4 we have shown that the $\mathbf{H}$ metamodel has been developed up to a Reynolds number equal to $100$. This limit has been derived from the data and it was not fixed a priori. To estimate this limit we have checked the direct comparison of $F^m$ and $F^M$ since it is a fair estimation of the correctness of the hypothesis behind our closure model. For the geometry and range of porosities tested, the correct limit is around Reynolds number $100$. After this limit the error between the two quantities starts to be appreciable and so the closure problem \eqref{eq:M_problem} is no more correct. We suppose that at higher Reynolds numbers the linear correlation hypothesis between the average fields and the perturbations (equations \eqref{eq:closure_viab_h1} and \eqref{eq:closure_viab_h2}) does not hold.
This chapter has been the basis for an article that has been already submitted and is now under review.

The interface treatment, based on the penalization method, has been investigated in chapter 5. It has been shown that the double linear smoothing of porosity and permeability has a positive effect on the correctness of the homogenized model. We have also shown that linear porosity profile derives directly from the geometry of the porous media and it is exact. On the contrary the linear smoothing for the permeability tensor is purely heuristic, but it can be supported by the fact that the porosity effects are largely the most important effect in the variability of $\mathbf{H}$. So, it is possible to argue that the two fields should have the same interface treatment. Another confirmation for this fact comes from the metamodel that, if left "free", it returns the linear profile of permeability at the interface without imposing it a priori. A paper is actually under preparation on the topics described in this chapter.
 
The VANS approach has been adopted in cases that naturally develop separation. The inclusion of a porous media layer has been tested and the solver has shown good computational performance without convergence problems. However, the physics of the separation is not much modified by the porous layer, as a matter of fact the recirculation bubble size remains almost the same. This results suggest that the laminar suppression mechanism could be not as effective as the turbulent one (already observed in literature). In any case, more simulations with different problem geometries are needed to generalize the results.

 The OpenFoam implementation of the macroscopic solver based on the VANS equations can be downloaded from github from the address: \url{https://github.com/appanacca/porous_solvers_OF}.
The code listing is not directly shown in the manuscript since detailing the solver implementation would have required to explain and describes in details many OpenFoam library 
technicalities. These details have been already addressed in multiple sources (\citet{jasak1996error}, \citet{moukalled2016finite} and 
\citet{maric2014openfoam}) and they are out of the scope of this work.
To someone not new to OpenFoam programming the comments inside the code listing are sufficient to clarify the technical points.


%\end{itemize}


\subsection{Possible future research extensions}

%\begin{itemize}

 The database from which we have built our metamodel for the tensor $\mathbf{H}$ can be extended. For example, we could easily include more data points to make the model more reliable. Another interesting part could be the extension to other fibers geometry section or even other completely different porous media geometries (spheres, rocks ...). The database could also be extended to moving porous media, the input parameters could include some of the typical dynamical parameters like the mass ratio or the stiffness of the fibers. New metamodelling approaches could also be explored. Especially, when the database grows, \textit{deep neural networks} could perform better than Kriging.

 The validation of the interface treatment requires more data from DNS simulations or experiments in similar configurations. The availability of high resolution data is still a missing piece in the field.

 The application of the macroscopic model to separated flow is only a starting point. We have shown that our model is able to provide fairly correct homogenized flow fields at a low computational cost. However, the capacity of a porous media layer to suppress the separation is still questioned, at least with the parameters used. This means that the optimal parameters are still to be found. An optimization procedure using the adjoint equation could solve this problem, now that we have clarified the penalization approach used in the VANS equations.

 Another possible extension to the metamodel could be the implementation of the algorithm \ref{algo:vans_elastic} (shown in section \ref{ph:moving}) for elastic porous media. Since the VANS solver is already been implemented this extension, for example by using a Bernoulli beam, should be fairly easy.
%\end{itemize}

